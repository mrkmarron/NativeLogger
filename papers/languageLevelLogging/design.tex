% !TeX root = LanguageLevelLogging.tex
This section describes opportunities, using language, runtime, or compiler support, to address 
general challenges surrounding logging outlined in \autoref{sec:intro}. We can roughly divide 
these into two classes -- performance oriented and functionality oriented. 

\subsection{Logging Performance}
\label{subsec:performancedesign}

\begin{design}
The cost of a disabled logging statement, one that is at a logging level that is not enabled, should 
have zero-cost at runtime. This includes both the direct cost of the logging action and the indirect cost 
of of building a format string and processing any arguments. 
\end{design}

For logging frameworks that are included as libraries the compiler/JIT does have a deep understanding 
of the enabled/disabled semantics of the logger and, in general, will not be able to fully eliminate 
the dead-code associated with disabled logging statements.





The major issues this proposal is intended to address are:

\noindent
Lack of high performance logging primitives and fundamental logging challenges.
\begin{enumerate}
 \item Cost of writing data to the log -- particularly with data formatted via 
    `util.inspect` and info such as timestamps.
 \item Ongoing tension between log detail when triaging issues and cost of logging 
    large amounts of 'uninteresting' data.
\item Parasitic costs of disabled logging statements which still execute code to 
    generate dead logging data (e.g., constructing unused strings).
\item Compiler optimizations of both enabled and disabled logging statements.
\end{enumerate}

\subsection{Logging Functionality}
\label{subsec:functionalitydesign}

\noindent
Challenges integrating log data from different sources and difficulty in post processing.
\begin{enumerate}
\item Difficulty in specifying uniform and appropriate logging levels across 
    multiple packages -- and quite possibly multiple logging frameworks.
\item Difficulty in ensuring all logging data is written to a consistent location 
    across multiple packages -- and quite possibly multiple logging frameworks.
\item Correlating fundamental information such as  
    transaction ids, such as async context and HTTP requests, and to log relevant 
    events in the core libraries with user logging data -- same for performance info.
\end{enumerate}

